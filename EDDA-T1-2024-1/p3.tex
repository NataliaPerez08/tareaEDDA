\section*{Pregunta 3}
\noindent Se pueden utilizar las estructuras de búsqueda de rangos ortogonales para determinar si un punto particular $(a,b)$ está en un conjunto dado, haciendo una consulta al rango $[a:a] \times [b:b]$.
\begin{enumerate}
   \item Prueba que hacer una consulta así en un árbol $KD$ toma tiempo $O(\log n)$.
   \item ¿Cuál es la complejidad para una consulta así en un árbol de rangos?
\end{enumerate}

\subsection*{Respuesta}
\begin{enumerate}
   \item Prueba que hacer una consulta así en un árbol $KD$ toma tiempo $O(\log n)$.
   
   El algoritmo de busqueda para el kd-tree es
   
   \begin{algorithmic}[1]
      \State Si $v$ es una hoja
         \State Regresa si $v$ esta en $R$
      
      \State De otra manera:
      \State Si la región $lc(v)$ (del hijo izquierdo de $v$) está contenida en $R$ entonces: 
               recorrer el subárbol izquierdo de $V$ y regresa los puntos almacenados.

      \State De otra manera:
      \State Si la región $lc(v)$ intersecta $R$ entonces: 
               busca en el subarbol $lc(v)$ en $R$
      \State Si la región $rc(v)$ esta contenida en $R$ entonces:
               recorrer el subarbol derecho de $v$ y regresa los puntos almacenados

      \State De otra manera:
      \State Si la región $rv(v)$ intersecta $R$ entonces:
               busca en el subarbol $rc$ en $R$
       
      \end{algorithmic}
      De forma que para la consulta del punto $(a,b)$ el algoritmo verifica en que subarbol esta por 4 y 7 y desciende por 5 u 8 sobre el subarbol izquierdo o derecho respectivamente, hasta llegar a una hoja. De manera que recorre la altura del árbol para hallar el punto una vez y como la altura del árbol es $\log n$. Toma $O(\log n)$ hacer una consulta de un punto particular $(a,b)$ al rango $[a:a] \times [b:b]$ en un árbol $KD$ 
  
      \item ¿Cuál es la complejidad para una consulta así en un árbol de rangos?
      
      El algoritmo de busqueda para el árboles de rango requiere de dos rutinas:

      EncuentraNodoDivisor$(\tau,x,x')$:
      \begin{algorithmic}[1]
         \State $v \gets raiz(T)$

         \While{$v$ no es una hoja y $x' \leq x_v$ o $x > x_v$}


         \If{$x'\leq x_v$}
            \State $v \gets lc(v)$
            \Else 
            \State $v \gets rc(v)$
         \EndIf
         \EndWhile
         \end{algorithmic}

      Algoritmo2DRangeQuery($\tau, [x:x']\times [y:y']$)
      
      \begin{algorithmic}[1]
         \State $v_x \gets$ EcuentraNodoDivisor$(\tau,x,x')$


         \If{$v_s$ es una hoja}
            \State Verifica si el punto en $v_s$ debe ser reportado.
            \Else 
            \State Sigue el camino a $x$ y realiza la busqueda en una 1 dimension en los subarboles derechos del camino.
            \State $v \gets lc(v_s)$

            \While{$v$ no es una hoja}
               \If{$x \leq x_v$}
               \State Realiza una busqueda de una dimension sobre $\tau_{assoc} (rc(v)),[y:y']$

               \State $v\gets lr(v)$
               \Else
               \State $v \gets rc(v)$ 
               \EndIf
            \EndWhile
            \State Verifica si el punto debe ser reportado
            \State De forma similar sigue el camino de $rc(v_s)$ a $x'$, haz la busqueda en una dimension sobre $[y:y']$ en las estructuras asociadas de los subarboles deñ camino y verifica si el punto almacenado en la hoja donde el camino termina debe ser reportado.
         \EndIf
      \end{algorithmic}
      Entonces en la consulta de un punto $(a,b)$ en $[a:a]\times[b:b]$ por 1 busca el nodo divisor que contiene a $a$ lo que toma $O(\log n)$ (la altura del árbol principal), si el nodo que contiene $a$ es una hoja entonces lo regresa. Si no es una hoja entonces realiza una busqueda sobre $\tau_{assoc}$ (estructura asociada) en el rango $[y:y']$ lo que toma $O(k)$ donde $k=1$ porque en el rango solo hay un punto.

      Por tanto la consulta de un punto $(a,b)$ en $[a:a]\times[b:b]$ es $O(\log n + 1) \rightarrow O(\log n)$

\end{enumerate}
\bigskip
