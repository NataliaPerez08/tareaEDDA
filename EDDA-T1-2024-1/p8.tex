\section*{Pregunta 8}
\noindent En algunas aplicaciones solo nos interesa el número de puntos que caen dentro de un rango y no reportar cada uno de ellos.
En este caso nos gustaría evitar el término $O(k)$ en el tiempo de consulta.

\begin{enumerate}
  \item Describe cómo un árbol de rangos de una dimensión puede adaptarse para que una consulta así se pueda realizar en tiempo $O(\log n)$.
  \item Usando la solución al problema para una dimensión, describe cómo se pueden responder consultas de conteo en rangos de $d$ dimensiones en tiempo $O(\log^d n)$.
  \item Describe cómo se puede usar la técnica de cascada para mejorar el tiempo de consulta en un factor $O(\log n)$ para dos y más dimensiones.
\end{enumerate}

\subsection*{Respuesta}

\begin{enumerate}
  \item Describe cómo un árbol de rangos de una dimensión puede adaptarse para que una consulta así se pueda realizar en tiempo $O(\log n)$.
  
  Dado un rango $[x,x']$ buscar en el árbol de rangos el punto $x$ y el punto $x'$. Luego buscar el nodo divisor $v_{split}$ utilizando el algoritmo presentado en el ejercicio 3.
  Por cada nodo en el camino de $x$ revisar si el punto almacenado en la nodo debe ser reportado (se encuentra en el intervalo). De manera similar sigue del camino a $x'$, reporta los puntos en los subarboles a la izquierda de del camino, y revisa si el punto almacenado en la hoja donde el camino termina debe ser reportado.

  Como el árbol de rangos se encuentra balanceado los caminos que recorre hasta las hojas tienen longitud $O(\log n)$, luego reportar (y contar) todos los nodos toma $O(n)$, por tanto la complejidad final del algoritmo es $O(\log n)+O(n) \rightarrow O(\log n)$
  
  \item Usando la solución al problema para una dimensión, describe cómo se pueden responder consultas de conteo en rangos de $d$ dimensiones en tiempo $O(\log^d n)$.
  Se repite la estrategia anterior sobre las estructuras auxiliares. 

  Sea $range = [x:x']\times [y: y'] \times [z:z'] \times \dots$ el rango que nos interesa consultar. Sea un árbol de rango $T$, con $d$ dimensiones, este tendrá nodos que almacenan listas con $(a_1,a_2,\dots,a_d)$, de manera que obtener todos puntos que cumplan la condicion $a_i \in range$ (por ejemplo $a_i \in [x:x']$) nos tomara $O(\log n)$ utilizando el algoritmo anterior, y si deseamos obtener los puntos $a_{i+2}$ que cumplen estar en el rango siguiente, es necesario revisar su estructura asociada cuya longitud es $O(\log n)$ utiliza el algoritmo de $6.a$ de manera que para $d=2$ se debe ejecutar dos veces el algoritmo por tanto su complejidad es $O(\log n * \log n)= O(\log^2)$ (Si solo se cuentan los puntos en el rango). Siguiendo de esta manera para $d=3$ se debe ejecutar tres veces el algoritmo por tanto su complejidad es $O(\log n * \log n * \log n)= O(\log^3)$. Y podemos concluir que para $d$ dimensiones se debe ejecutar el algoritmo $d$ veces el algoritmo y su complejidad es $O(\log ^d n)$

  \item Describe cómo se puede usar la técnica de cascada para mejorar el tiempo de consulta en un factor $O(\log n)$ para dos y más dimensiones.
  
  De manera similar al indice anterior seleccionar primero los puntos cuya cuya coordenada $d-$ésima está en el rango correcto en subconjuntos canónicos $O(log n)$, y luego resolviendo una consulta $(d-1)$-dimensional en estos subconjuntos. 
\end{enumerate}

\section*{Pregunta 9} Describe el proceso de dinamizar el árbol de rangos de dimensión 3.

La técnica consiste en crear bosques de árboles utilizando bits que nos indicara el tamaño del árbol:
$$2^0  = 1$$ 
$$2^1  = 2 $$
$$2^2  = 4 $$
$$2^3  = 8$$
$$2^4  = 16$$
$$2^5  = 32 $$
$$2^6  = 64 $$
$$2^7  = 128 $$
$$2^8  = 256 $$
$$2^9  = 512$$
$$2^{10} = 1024 $$
Y las consultas se realizaran sobre cada una de ellas.
\bigskip
